\documentclass[11pt]{article}
% \setmainfont[Ligatures=TeX]{Linux Libertine O}

\usepackage{booktabs}
\usepackage[margin=10pt,font=small, textfont=it, labelfont=bf,
               labelsep=quad]{caption}
\usepackage{csquotes}
\usepackage{enumitem}
\usepackage{float}
\usepackage{hyperref}
\usepackage{longtable}
\usepackage{rotating}
\usepackage{sectsty}
\usepackage{subcaption}
\usepackage{wrapfig}
\usepackage[svgnames]{xcolor}

\usepackage[
a4paper, 
height = 24cm,
width = 16cm
]{geometry}

\usepackage{fancyhdr}
\usepackage[yyyymmdd,hhmmss]{datetime}
\pagestyle{fancy}
\lfoot{}
\cfoot{}
\rfoot{\thepage} %Version of \today, \currenttime}

\sectionfont{\MakeUppercase}
\subsectionfont{\MakeUppercase}

\usepackage{url}
\urlstyle{same}

\usepackage{graphicx}
\graphicspath{{./media/}}
\usepackage[UKenglish]{babel}
\usepackage{fontspec}

\setmainfont[Ligatures=TeX]{Charter}

\usepackage[xcolor]{mdframed}
\mdfdefinestyle{theoremstyle}{%
  % linecolor=red,linewidth=2pt,%
  backgroundcolor=PowderBlue,
  frametitlerule=true,%
  frametitlebackgroundcolor=gray!20,
  innertopmargin=\topskip,
}
\mdtheorem[style=theoremstyle]{definition}{Data Assessment}

\usepackage[
backend=biber,
sortlocale=auto,
sorting=none,
style=numeric,
citestyle=numeric,
maxnames=4
]{biblatex}
\addbibresource{trans-refs.bib}

% \usepackage[colorinlistoftodos,prependcaption,textsize=footnotesize]{todonotes}
% \presetkeys%
%     {todonotes}%
%     {inline,backgroundcolor=yellow}{}

% correct bad hyphenation here
%\hyphenation{op-tical net-works semi-conduc-tor}

\newcommand{\term}[1]{\textit{#1}}
\newcommand{\project}{\textit{Data and Design for Property Planning}}

\newenvironment{qt}
{``\itshape }
{''\ignorespacesafterend\ }

\title{Transduction of Discourse Representations}
\date{January 1983}
\author{Ewan Klein}

\begin{document}
\maketitle

\section{Introduction}

Recent work in syntactic theory has indicated that many of the phenomena thought to require rules
with transformational power can in fact be insightfully handled by formal devices which generate
only context-free languages. The framework of Generalized Phrase Structure Grammar (GPSG) has
proved particularly successful in this respect, allowing precise and detailed analyses of a wide
range of syntactic facts (cf.\ Gazdar (1981), (1982), Gazdar, Pullum and Sag (forthcoming)).
Unbounded dependencies of the sort associated with the transformational rule of \textit{wh}-Movement would
seem to provide a demanding test for any syntactic approach that eschews movement rules. Yet they
have yielded gracefully to a phrase structure treatment in which the presence and position of `gaps'
in a structure are coded by a special set of syntactic categories, called \em{derived} (or \em{slash})
categories.

Roughly speaking, if $\alpha$ and $\beta$ are any basic categories (i.e. members of the
set $N$ of nonterminal symbols), then $\alpha$/$\beta$ is a derived category. The interpretation of a
derived category is as follows: A node labelled $\alpha$/$\beta$ will dominate just the trees that
can be dominated by $\alpha$, except that somewhere in the $\alpha$/$\beta$ tree there must be a node of
the form $\beta$/$\beta$ dominating the empty string, and every node linking $\alpha$/$\beta$ and
$\beta$/$\beta$ must be of the form $\sigma$/$\beta$. That is, $\alpha$/$\beta$ labels a node of
type a which dominates material containing a gap of the type $\beta$. For example, S/NP is a
sentence which has an NP missing somewhere (not necessarily from within that immediate sentence);
There are also derived rules, which allow the propagation of a gap through a tree, and linking rules
which allow the introduction of a category with a gap. For example, given rule (1), we get the
derived rules (2) and (3).


\end{document}
